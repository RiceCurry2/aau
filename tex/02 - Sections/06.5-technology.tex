\chapter{Technology}

\section{Sensors}

We already provided information on less advanced methods for mine detection, see ‘Primitive demining methods’, however more advanced technology such as the use of detection sensors could provide a safer alternative. These sensors can be divided into the following groups \cite{HumanitarianDemining2017}:
\begin{itemize}
\setlength{\itemsep}{0.05\baselineskip}
	\item Explosive detectors
	\item Electromagnetic sensors
	\item Electro-optic sensors
\end{itemize}

Explosive detectors are used to detect the explosive material of a mine. Electromagnetic sensors, such as metal detectors (MD), which are used to measure magnetized metals, or ground penetrating radar (GPR), which can identify plastic. An electro-optic sensor such as an infrared sensor (IR), uses light waves to measure the distance to some objects in the soil. All these could potentially be used to locate mines \cite{HumanitarianDemining2017}.

Limitations on these sensors do exist, which is why a combination of multiple sensors might be the best option for a mine detection robot. For instance, an electro-optic sensor such as an infrared sensor cannot be used to centre specific mines. This means they are only capable of scanning an area. Ground-penetrating radar is relying on the ground condition to determine the depth of its ground penetration but can however detect other materials than metal beneath the ground. Metal detectors might be cheap and reliable but can, as the name implies, only detect metal \cite{HumanitarianDemining2017}.
So in general, making a robot effective, means using a few different sensors and detectors to cover the limitations of the single one \cite{6LeggedRobot2007}.



