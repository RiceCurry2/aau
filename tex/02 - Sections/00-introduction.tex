\chapter{Introduction} \label{chap:Introduction}

Landmine are encased explosive charges either buried just below the surface of the ground, buried deep under ground, or placed above ground hidden in bushes, grass, etc.\\
Landmines generally have more or less simple trigger mechanisms, such as \cite{EB:Mine}:

\begin{itemize}
\setlength{\itemsep}{0.05\baselineskip}
    \item Pressure censors; reacting when weight of persons, vehicles, cattle, etc. exceeds a preset limit
    \item Tripwires; reacting when the wire is pulled or stepped on
    \item Remote control; manually detonating the mine from a remote position
    \item Timers; detonating the mine after a preset amount of time
\end{itemize}

The versatility and simplicity of landmines in combination with the strategic advantage of being able to effect the enemy, without tying own troops in the operation has made landmines a popular weapon to use in times of conflict and war throughout history.

Landmines are cheap to manufacture, and therefore they have been produced and deployed in great numbers since their invention. As a result, many active mines remain buried/hidden throughout the world. Presently, it is estimated that 59 countries are still affected by landmines. \cite{LandmineMonitor2019}

However, statistics also show that the work done throughout the past 20 years has resulted in 2880 square kilometres of land being cleared of landmines, and more than 4,6 million anti-personnel mines being destroyed. It is estimated that there are more than 2000 square kilometres of landmine contaminated land remaining thru out the world \cite{clearingTheMines}.\\

The purpose of landmines range from preventing an opponent access to critical installations or areas of strategic value, to leading the enemy a specific way on the battlefield, to simply instilling fear and demoralizing the enemy. Furthermore the minefield, also functions as an alarm system, because of the loud noise created by the explosions \cite{clearingTheMines}.\\

A heavy downside to mines is that they do not distinguish between military opponents, civilians or even animals.

This means that the usefulness of mines in times of war turns into a problem for the country and especially its population in the aftermath of the war. Landmines restrict access to parts of the country and limit the the population in utilizing agricultural land, roads, etc.

Landmines further effect the population, as many civilians fall victim to them each day - ether suffering death or serious injuries such as dismemberment or organ damages caused by fragments  \cite{clearingTheMines}\\.

Over 20 years ago countries started a joint effort to rid the world of landmines. This involved putting more focus on the issue and laying in the financial resources needed to start the work of locating and destroying the landmines, which is also known as demining \cite{clearingTheMines}.

This joint effort is known as the Ottawa Convention, or formally: the Anti-Personnel Mine Ban Treaty. The treaty was drafted under the UN in 1997 and put into motion in 1999 \cite{treatyICBL}.\\

\noindent The International Mine Action Standards, also known as IMAS, defines demining as such:\\

“\textit{Activities which lead to the removal of Explosive Ordnance hazards, including technical survey, mapping, clearance, marking, post-clearance documentation, community mine action liaison and the handover of cleared land. Demining may be carried out by different types of organizations, such as NGOs, commercial companies, national mine action teams or military units. Demining may be emergency-based or developmental.}” \cite{IMAS}\\

\noindent IMAS also states that humanitarian demining and demining are interchangeable terms \cite{IMAS}.\\

Demining still relies heavily on manual labour, which is very time demanding and involves a big risk of injury to the workers.

However, over the years a lot of advances have been made to speed up this process and reduce the risk to the workers. Some of these include heavy reinforced vehicles capable of withstanding explosions, intended to simply trigger the landmines by various mechanisms.

More recently these advances has had an increasing focus on semi-automated and fully automated systems, presently leading to the development of demining robots. Demining robots such as the MineWolf have been developed in order to remove mines in a way, that limits or excludes the human involvement in the direct detection and removal of the mines \cite{mineWolf}.
