\addcontentsline{toc}{chapter}{Title page}
\thispagestyle{empty}
\pdfbookmark[0]{English title page}{label:titlepage_en}
\aautitlepage{%
  \englishprojectinfo{
    Mobile Robot for Localization of Landmines %title
  }{%
    Reality and Models %theme
  }{%
    ROB1 P1 2nd half of 1st Semester %project period
  }{%
    B332-b % project group
  }{%
    %list of group members
    Jesper Poscholann Hammer\\
    Maiken Cecilie Lanng\\
    Kjartan Nolsøe Jespersen\\ 
    Kasper Maarschalk Hytting
  }{%
    %list of supervisors
    Matthias Rehm
  }{%
    1 % number of printed copies
  }{%
    \today % date of completion
  }%
}{%department and address
  \textbf{The Technological Faculty for IT and Design}\\
  Department of Electronic Systems\\
  Niels Jernes Vej 10, 9220 Aalborg Øst\\
  \href{http://www.aau.dk}{http://www.aau.dk}
}{% the abstract
Landmines are a major risk to civilian populations in affected areas. It is estimated that 59 countries still are influenced by landmines and other explosives. This provides a financial burden for governments and a health risk for the population in the proximity of these mine areas.

The report sheds light on the devastating amount of casualties due to landmines trough out the world and it is soon revealed that Afghanistan has the highest death count, hence this report will have a slight focus on that country.\\
    
The process of finding and removing landmines is called demining. This report mainly focuses on humanitarian demining, because of the needed assistance in making this easier, safer, and faster. Current humanitarian demining is a dangerous endeavor and therefore development in robotics to assist in demining is being made. These robots vary a lot in their specifications and are still not properly implemented in the demining process. This could be because of the cost, the robots’ specialization in singular types of mines, or the advanced systems and mechanics. Some of the current robots will be presented in this report.
}